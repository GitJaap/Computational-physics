\documentclass[twoside]{article} 
\usepackage{pgfplots} 
\usepackage{amsmath}
\usepackage{amsfonts}
\usepackage{graphicx}
\usepackage[procnames]{listings}
\usepackage{color}
\usepackage{lipsum} % Package to generate dummy text throughout this template

\usepackage[sc]{mathpazo} % Use the Palatino font
\usepackage[T1]{fontenc} % Use 8-bit encoding that has 256 glyphs
\linespread{1.05} % Line spacing - Palatino needs more space between lines
\usepackage{microtype} % Slightly tweak font spacing for aesthetics

\usepackage[hmarginratio=1:1,top=32mm,columnsep=20pt]{geometry} % Document margins
\usepackage{multicol} % Used for the two-column layout of the document
\usepackage[hang, small,labelfont=bf,up,textfont=it,up]{caption} % Custom captions under/above floats in tables or figures
\usepackage{float} % Required for tables and figures in the multi-column environment - they need to be placed in specific locations with the [H] (e.g. \begin{table}[H])
\usepackage{hyperref} % For hyperlinks in the PDF

\usepackage{lettrine} % The lettrine is the first enlarged letter at the beginning of the text
\usepackage{paralist} % Used for the compactitem environment which makes bullet points with less space between them

\usepackage{abstract} % Allows abstract customization
\renewcommand{\abstractnamefont}{\normalfont\bfseries} % Set the "Abstract" text to bold
\renewcommand{\abstracttextfont}{\normalfont\small\itshape} % Set the abstract itself to small italic text

\usepackage{titlesec} % Allows customization of titles
\renewcommand\thesection{\Roman{section}} % Roman numerals for the sections
\renewcommand\thesubsection{\Roman{subsection}} % Roman numerals for subsections
\titleformat{\section}[block]{\large\scshape\centering}{\thesection.}{1em}{} % Change the look of the section titles
\titleformat{\subsection}[block]{\large}{\thesubsection.}{1em}{} % Change the look of the section titles

\usepackage{fancyhdr} % Headers and footers
\pagestyle{fancy} % All pages have headers and footers
\fancyhead{} % Blank out the default header
\fancyfoot{} % Blank out the default footer
\fancyhead[C]{Computational Physics: Pairwise interaction model of an FCC packed material and the verification of theoretically proved properties by simulation $\bullet$ February 2016} % Custom header text
\fancyfoot[RO,LE]{\thepage} % Custom footer text

\usepackage{subcaption}
\captionsetup{compatibility=false}

%----------------------------------------------------------------------------------------
%	TITLE SECTION
%----------------------------------------------------------------------------------------

\title{\vspace{-15mm}\fontsize{18pt}{10pt}\selectfont\textbf{Pairwise interaction model of an FCC packed material and the verification of theoretically proved properties by simulation}} % Article title

\author{
\large
\textsc{Jaap Wesdorp}, $\hspace{10pt}$ \textsc{Bas Dirksen} \\ % Your name
\normalsize $^\dagger$Delft University of Technology \\ % Your institution
\normalsize \href{mailto:john@smith.com}{b.dirksen@student.tudelft.nl}\\ \normalsize \href{mailto:john@smith.com}{j.j.wesdorp@student.tudelft.nl} % Your email address
\vspace{-5mm}
}
\date{}

%----------------------------------------------------------------------------------------

\begin{document}
	
\definecolor{keywords}{RGB}{255,0,90}
\definecolor{comments}{RGB}{0,0,113}
\definecolor{red}{RGB}{160,0,0}
\definecolor{green}{RGB}{0,150,0}

\lstset{language=Python, 
	basicstyle=\ttfamily\small, 
	keywordstyle=\color{keywords},
	commentstyle=\color{comments},
	stringstyle=\color{red},
	showstringspaces=false,
	identifierstyle=\color{green},
	procnamekeys={def,class}}

\maketitle % Insert title

\thispagestyle{fancy} % All pages have headers and footers

%----------------------------------------------------------------------------------------
%	ABSTRACT
%----------------------------------------------------------------------------------------

\begin{abstract}

\noindent  Abstract here

\end{abstract}

%----------------------------------------------------------------------------------------
%	ARTICLE CONTENTS
%----------------------------------------------------------------------------------------

\section{Introduction}

\lettrine[nindent=1em,lines=3]{T} \ \ \ he introduction starts here

%------------------------------------------------

\section{Methods}
\subsection{Problem Formulation}
In this paper, we consider 
\section{Results and Discussion}
\subsection{Results}

\section{Conclusions}
This study shows that 
\clearpage

\section{Appendix A - Python Code}\vspace{1em}
\subsection{Main program}
\begin{lstlisting}
%%python CODE HIER
for i in range(0,10):
	do cool stuff
	
\end{lstlisting}\vspace{1em}\clearpage


\section{Appendix B}


\clearpage

\begin{thebibliography}{1}
%%random voorbeeld boeken
\bibitem{ref1}
Baron, S. (1996).
\newblock Medical Microbiology, 4th edition
\newblock {\em University of Texas Medical Branch}, Ch.17

\bibitem{ref2}
Carreau, A., El Hafny-Rahbi, B., Matejuk, A., Grillon, C., Kieda, C. (2011).
\newblock Why is the partial oxygen pressure of human tissues a crucial parameter? Small molecules and hypoxia.
\newblock {\em J Cell Mol Med}, 15:1239-53
\end{thebibliography}

\end{document}

















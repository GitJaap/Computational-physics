\documentclass[twoside]{article} 
\usepackage{pgfplots} 
\usepackage{amsmath}
\usepackage{amsfonts}
\usepackage{graphicx}
\usepackage[procnames]{listings}
\usepackage{color}
\usepackage{lipsum} % Package to generate dummy text throughout this template

\usepackage[sc]{mathpazo} % Use the Palatino font
\usepackage[T1]{fontenc} % Use 8-bit encoding that has 256 glyphs
\linespread{1.05} % Line spacing - Palatino needs more space between lines
\usepackage{microtype} % Slightly tweak font spacing for aesthetics

\usepackage[hmarginratio=1:1,top=32mm,columnsep=20pt]{geometry} % Document margins
\usepackage{multicol} % Used for the two-column layout of the document
\usepackage[hang, small,labelfont=bf,up,textfont=it,up]{caption} % Custom captions under/above floats in tables or figures
\usepackage{float} % Required for tables and figures in the multi-column environment - they need to be placed in specific locations with the [H] (e.g. \begin{table}[H])
\usepackage{hyperref} % For hyperlinks in the PDF

\usepackage{lettrine} % The lettrine is the first enlarged letter at the beginning of the text
\usepackage{paralist} % Used for the compactitem environment which makes bullet points with less space between them

\usepackage{abstract} % Allows abstract customization
\renewcommand{\abstractnamefont}{\normalfont\bfseries} % Set the "Abstract" text to bold
\renewcommand{\abstracttextfont}{\normalfont\small\itshape} % Set the abstract itself to small italic text

\usepackage{titlesec} % Allows customization of titles
\renewcommand\thesection{\Roman{section}} % Roman numerals for the sections
\renewcommand\thesubsection{\thesection.\Roman{subsection}} % Roman numerals for subsections
\titleformat{\section}[block]{\large\scshape\centering}{\thesection.}{1em}{} % Change the look of the section titles
\titleformat{\subsection}[block]{\large}{\thesubsection.}{1em}{} % Change the look of the section titles

\usepackage{fancyhdr} % Headers and footers
\pagestyle{fancy} % All pages have headers and footers
\fancyhf{}
\fancyhead[C]{Molecular Dynamics Simulation of a Lenard-Jones Interacting Molecule for Obtaining Macroscopic Static Physical Properties $\bullet$ February 2016} % Custom header text
\fancyfoot[RO,LE]{\thepage} % Custom footer text
\fancyfoot[CO,CE]{Jaap Wesdorp \& Bas Dirkse}
\fancypagestyle{firststyle}
{	
	\fancyhf{}
	\renewcommand{\headrulewidth}{0pt}
   	\fancyfoot[RO,LE]{\thepage}
   	\fancyfoot[CO,CE]{Jaap Wesdorp \& Bas Dirkse}
}

\usepackage{subcaption}
\captionsetup{compatibility=false}

%----------------------------------------------------------------------------------------
%	TITLE SECTION
%----------------------------------------------------------------------------------------

\title{\vspace{-15mm}\fontsize{18pt}{10pt}\selectfont\textbf{Molecular Dynamics Simulation of a Lenard-Jones Interacting Molecule for Obtaining Macroscopic Static Physical Properties}} % Article title

\author{
\large
\textsc{Jaap Wesdorp}$^\dagger$, $\hspace{10pt}$ \textsc{Bas Dirkse}$^\dagger$ \\ % Your name
\normalsize $^\dagger$Delft University of Technology \\ % Your institution
\normalsize \href{mailto:j.j.wesdorp@student.tudelft.nl}{j.j.wesdorp@student.tudelft.nl} \\
\normalsize \href{mailto:b.dirkse@student.tudelft.nl}{b.dirkse@student.tudelft.nl} 
}
\date{\today\vspace{-8mm}}

%----------------------------------------------------------------------------------------

\begin{document}
	
\definecolor{keywords}{RGB}{255,0,90}
\definecolor{comments}{RGB}{0,0,113}
\definecolor{red}{RGB}{160,0,0}
\definecolor{green}{RGB}{0,150,0}

\lstset{language=Python, 
	basicstyle=\ttfamily\small, 
	keywordstyle=\color{keywords},
	commentstyle=\color{comments},
	stringstyle=\color{red},
	showstringspaces=false,
	identifierstyle=\color{green},
	procnamekeys={def,class}}

\maketitle % Insert title
\thispagestyle{firststyle} % Only footer on first page

%----------------------------------------------------------------------------------------
%	ABSTRACT
%----------------------------------------------------------------------------------------

\begin{abstract}
\noindent  \lipsum[1]

\end{abstract}

%----------------------------------------------------------------------------------------
%	ARTICLE CONTENTS
%----------------------------------------------------------------------------------------

\section{Introduction}

\lettrine[nindent=1em,lines=2]{I}
n this paper we intend to derive experimental macroscopic properties of a material from a microscopic description of the molecular interactions. In statistical physics many macroscopic quantities of many-particle systems can be found as an ensemble average over the possible microscopic states. Any practical macroscopic system consists of so many possible microscopic states that it is infeasible to average over all of the possible states computationally. But given a large subset of the possible states, we may assume that physical quantities averaged over the subset are close to the ensemble average. In molecular dynamics (MD) we initialize a specific state determined by some system parameters and let it evolve in time, traversing along its physical trajectory in the phase space as determined by the equations of motion. We therefore generate a large subset of possible states which are correlated in time. Using appropriate averaging over time we can obtain estimates of the ensemble average and therefore the relevant physical quantities.

We restrict ourselves to studying static physical properties of the system at equilibrium, although MD could also be used to study dynamical properties of a system. We carry out simulations for Argon, which is studied extensively in the literature and is modeled easily using the Lenard-Jones interaction potential. First we compute the heat capacity and compare it to theoretical results in the case of a hot and dilute gas or a cold and dense solid to verify the result. Next we compute the pressure as a function of the density at different temperatures and compare this with experimental results. Finally we compute the pair correlation function and comment on its qualitative behavior.

%------------------------------------------------

\section{Methods}
\subsection{Molecular Dynamics and the interaction potential for Argon}
Relevant topics in order are:
\begin{enumerate}
\item Interaction potential
\item Equation of motion
\item Discretization: Verlet algorithm
\item Boundary conditions
\item Initial conditions
\end{enumerate}

\subsection{Computation of physical quantities}
\begin{enumerate}
\item Computation of the heat capacity
\item Computation of the pressure
\item Computation of the pair correlation function
\end{enumerate}


\section{Results and Discussion}
\subsection{Results}
\lipsum[5-9]
\section{Conclusions}
\lipsum[4]

%\section{Appendix A - Python Code}\vspace{1em}
%\subsection{Main program}
%\begin{lstlisting}
%%%python CODE HIER
%for i in range(0,10):
%	do cool stuff
%	
%\end{lstlisting}\vspace{1em}\clearpage
%
%
%\section{Appendix B}


\begin{thebibliography}{1}
%%random voorbeeld boeken
\bibitem{ref1}
Baron, S. (1996).
\newblock Medical Microbiology, 4th edition
\newblock {\em University of Texas Medical Branch}, Ch.17

\bibitem{ref2}
Carreau, A., El Hafny-Rahbi, B., Matejuk, A., Grillon, C., Kieda, C. (2011).
\newblock Why is the partial oxygen pressure of human tissues a crucial parameter? Small molecules and hypoxia.
\newblock {\em J Cell Mol Med}, 15:1239-53
\end{thebibliography}

\end{document}
















